\documentclass[10pt,a4paper]{article}
\usepackage[utf8]{inputenc}
\usepackage{amsmath}
\usepackage{amsfonts}
\usepackage{amssymb}
\usepackage{graphicx}
\usepackage[margin=1.25in]{geometry}
\usepackage{hyperref}

\newcommand{\rmd}{\mathrm{d}}
\newcommand{\dif}[2]{\frac{\rmd #1}{\rmd #2}}
\newcommand{\difn}[3]{\frac{\rmd^{#3} #1}{\rmd #2^{#3}}}

\title{Waves}
\begin{document}

\part*{Lecture 4}

\subsubsection*{Comments on the wave equation}

\begin{itemize}
\item It's a one-dimensional wave equation.
\item It applies to left and right travelling waves.
\item It also applies to standing waves.
\end{itemize}

So,

\begin{align*}
y(x, t) = A e^{i(kx-\omega t)}
\end{align*}

This is a solution to the wave equation, and we find that:

\begin{align*}
v^{2} &= \frac{\omega^{2}}{k^{2}} \\
v &= \frac{\omega}{k}
\end{align*}

For light:

\begin{align*}
\difn{E}{x}{2} = \mu_{0} \epsilon_{0} \difn{E}{t}{2}
\end{align*}

where $E$ is the electric field, $\mu_{0}$ is the permeability of free space, and $\epsilon_{0}$ is the permittivity of free space.

\begin{align*}
v = \frac{1}{\sqrt{\mu_{0}\epsilon_{0}}}
\end{align*}

\section*{Examples of mechanical waves}

\subsection*{Waves on a taut string}

[Image]

Consider a transverse wave travelling left to right along a taut string. The individual particles of the string only move along the $y$ axis and not along the $x$ axis.

Consider a small element of the string.

[Image]

Tension in the string means that the forces are acting along the string itself.

\begin{align*}
F_{1, x} &= F_{2, x} = T \\
F_{1, y} &= F_{1, x} \tan \theta_{1} = F_{1, x} \dif{y}{x} \left[ x_{1} \right] \\
F_{2, y} &= F_{2, x} \tan \theta_{2} = F_{2, x} \dif{y}{x} \left[ x_{1} + \Delta x \right]
\end{align*}

There is a net upward force.

\begin{align*}
F_{y} = F_{2, y} - F_{1, y} = T \left( \dif{y}{x} \left[ x_{1} \right] - \dif{y}{x} \left[ x_{1} + \Delta x \right] \right)
\end{align*}

Newton's Second Law is

\begin{align*}
F = ma
\end{align*}

The mass of this small element of the string is given by

\begin{align*}
m = \mu \Delta x
\end{align*}

where $\mu$ is the mass per unit length.

\begin{align*}
F_{y} = \mu \Delta x \difn{y}{t}{2} &= T \left( \dif{y}{x} \left[ x_{1} \right] - \dif{y}{x} \left[ x_{1} + \Delta x \right] \right) \\
\mu \difn{y}{t}{2} &= \frac{ T \left( \dif{y}{x} \left[ x_{1} \right] - \dif{y}{x} \left[ x_{1} + \Delta x \right] \right) }{\Delta x} 
\end{align*}

Take the limit of $\Delta x \rightarrow 0$

\begin{align*}
\mu \difn{y}{t}{2} &= T \difn{y}{x}{2} \\
v &= \sqrt{\frac{T}{\mu}} \\
\end{align*}

\subsection*{Calculating how much energy is transmitted by a wave}
\subsubsection*{Transfer of power}

We know that waves transfer energy. It is useful to consider the power - the rate of energy transfer.

Consider a wave travelling from left to right again.

[Image]

We need to find out how much force is being applied to the piece of string to the right of point A.

\begin{align*}
F_{y} &= -F_{x} \dif{y}{x} \\
P &= F_{x} v \\
P &= -F_{x} \dif{y}{x} \dif{y}{t}
\end{align*}

This is instantaneous power transfer at time t and position x. For a sinusoidal wave:

\begin{align*}
y &= A \cos(kx - \omega t) \\
\dif{y}{x} &= -Ak \sin(kx - \omega t) \\
\dif{y}{t} &= A \omega \sin (kx - \omega t) \\
P(x, t) &= F_{x} A^{2} k \omega \sin^{2} (kx - \omega t)
\end{align*}

\end{document}