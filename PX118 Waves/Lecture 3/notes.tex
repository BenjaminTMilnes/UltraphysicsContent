\documentclass[10pt,a4paper]{article}
\usepackage[utf8]{inputenc}
\usepackage{amsmath}
\usepackage{amsfonts}
\usepackage{amssymb}
\usepackage{graphicx}
\usepackage[margin=1.25in]{geometry}
\usepackage{hyperref}

\newcommand{\rmd}{\mathrm{d}}
\newcommand{\dif}[2]{\frac{\rmd #1}{\rmd #2}}
\newcommand{\difn}[3]{\frac{\rmd^{#3} #1}{\rmd #2^{#3}}}

\title{Waves}
\begin{document}

\part*{Lecture 3}

\subsubsection*{Dispersion}

A typical wave:

\begin{align*}
u(x, t) = A \cos(kx - \omega t)
\end{align*}

Normally for waves in real media there will be a specific relationship between $\omega$ and $k$

\begin{align*}
\omega = f(k)
\end{align*}

This is known as the \textbf{dispersion relation}.

\paragraph*{Example: Sound Waves}

\begin{align*}
\omega = k \sqrt{\frac{B}{\rho}}
\end{align*}

where $B$ is the bulk modulus of air, and $\rho$ is the density of the air. So,

\begin{align*}
v = \frac{\omega}{k} = \sqrt{\frac{B}{\rho}}
\end{align*}

$\rightarrow \quad$ $v$ is independent of $\omega$ and $k$

$\rightarrow \quad \omega \propto k$

\vspace*{10pt}

The group velocity:

\begin{align*}
\frac{\partial \omega}{\partial k} &= \sqrt{\frac{B}{\rho}} \\
\text{i.e.} \quad v &= v_g
\end{align*}

Sound waves are therefore \textbf{non-dispersive}.

\paragraph*{Example: Deep Water Waves}

\begin{align*}
\omega &= \text{const.} \times \sqrt{gk} \\
v &= \text{const.} \times \sqrt{\frac{g}{k}}
\end{align*}

Therefore deep water waves \textbf{are dispersive}, as $v$ depends on $k$.

\begin{align*}
v_g &= \frac{\partial \omega}{\partial k} = \frac{\text{const.}}{2} \sqrt{\frac{g}{k}} = \frac{v}{2} \\
v_g &\neq v
\end{align*}

\subsubsection*{Brief comments about light}

We tend to think of light as travelling at $c$ ($3 \times 10^8 \mathrm{ms^{-1}}$). This is only really true in a vacuum, where:

\begin{align*}
\omega = ck \quad \text{non-dispersive}
\end{align*}

In most media the speed of light is less than $c$. For example, in glass:

\begin{align*}
v \approx \frac{2}{3} c
\end{align*}

A slightly odd example: electromagnetic waves in Earth's ionosphere:

\begin{align*}
\text{dispersion relation:} \quad \omega^2 = c^2 k^2 + \omega_p^2
\end{align*}

where $\omega_p$ is a constant - the frequency of natural circular oscillation of electrons in the ionosphere. So,

\begin{align*}
v &= \sqrt{c^2 + \frac{\omega_p^2}{k^2}} \\
v_g &= \frac{c^2 k}{\sqrt{c^2 k^2 + \omega_p^2}}
\end{align*}

Note that the velocity of the light waves is actually \textit{faster} than the speed of light, but that the group velocity is \textit{slower} than the speed of light, and it is the group velocity that matters here.

\vspace*{10pt}
[Image]
\vspace*{10pt}

The energy travels at $v_g$, and it is always the case that $v_g < c$.

\section*{The Wave Equation}

Take a general form for a travelling wave:

\begin{align*}
y(x, t) &= f(x - vt) \\
&= f(g(x, t))
\end{align*}

We're heading for a second order differential equation. Differentiation using the chain rule:

\begin{align*}
\dif{y}{x} &= \dif{f}{g} \dif{g}{x} \\
&= \dif{f}{g} \times 1 \\
\\
\dif{y}{t} &= \dif{f}{g} \times \dif{g}{t} \\
&= \dif{f}{g} \times -v \\
&= -v \dif{y}{x}
\end{align*}

Note, for a left-travelling wave:

\begin{align*}
\dif{y}{t} = + v \dif{y}{x}
\end{align*}

\begin{align*}
\difn{f(g)}{x}{2} = \difn{f}{g}{2} \times \left( \dif{g}{x} \right) ^2 + \dif{f}{g} \times \difn{g}{x}{2}
\end{align*}

We find that:

\begin{align*}
\difn{y}{x}{2} = \difn{f}{g}{2} \\
\difn{y}{t}{2} = v^2 \difn{f}{g}{2} \\
\\
\difn{y}{t}{2} = v^2 \difn{y}{x}{2}
\end{align*}

This is the Wave Equation.


\end{document}