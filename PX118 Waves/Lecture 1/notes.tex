\documentclass[10pt,a4paper]{article}
\usepackage[utf8]{inputenc}
\usepackage{amsmath}
\usepackage{amsfonts}
\usepackage{amssymb}
\usepackage{graphicx}
\usepackage[margin=1.25in]{geometry}
\usepackage{hyperref}

\title{Waves}
\begin{document}

\part*{Lecture 1}

\section*{Wave Phenomena}

There are three types of wave that we are interested in:

\begin{enumerate}
\item Mechanical Waves: waves on a string, waves in an organ pipe, waves on a rod, seismic waves, gravity waves
\item Electrical / Magnetic Waves:
\begin{enumerate}
\item In a material: alternating current of electrons (mains electricity), "spin waves" in magnetism
\item In free space (vacuum); propagation of electromagnetic waves; no disturbance of any material; light. Electromagnetic waves can travel through materials, e.g. glass, water.
\end{enumerate}
\item Matter Waves (de Broglie Waves)
\begin{enumerate}
\item Important for microscopic objects, e.g. electrons, neutrons.
\begin{align*}
\text{wavelength} = \lambda = \frac{h}{p}
\end{align*}
where $h$ is the Planck Constant, and $p$ is the momentum \\
$\rightarrow$ wave-particle duality \\
$\rightarrow$ foundation of quantum physics
\end{enumerate}
\end{enumerate}

\section*{Concepts and Definitions}

\subsection*{The Waveform}

In general, we have:

\begin{align*}
u(t) = f(x \pm ct)
\end{align*}

where $u$ is the displacement caused by the wave, $c$ is the speed of the wave, a minus sign indicates a \textbf{right-travelling} wave, and a plus sign indicates a \textbf{left-travelling} wave.

Consider a sinusoidal wave:

\begin{align*}
u(t) = A \sin(kx - \omega t)
\end{align*}

where $A$ is the amplitude of the wave, $k$ is the wavenumber $= \frac{2\pi}{\lambda}$, $\omega$ is the angular frequency $= 2\pi f$. The speed of the wave, $v$, is given by $\frac{\omega}{k}$, which is also called the \textbf{phase velocity}. We also define the \textbf{group velocity}, $v_g = \frac{\partial \omega}{\partial k}$.

For a light string, $v = \sqrt{\frac{T}{\mu}}$, where $T$ is the string tension, and $\mu$ is the mass per unit length.

\end{document}