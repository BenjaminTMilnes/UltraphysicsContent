\documentclass[10pt,a4paper]{article}
\usepackage[utf8]{inputenc}
\usepackage{amsmath}
\usepackage{amsfonts}
\usepackage{amssymb}
\usepackage{graphicx}
\usepackage[margin=1.25in]{geometry}
\usepackage{hyperref}

\title{Waves}
\begin{document}

\part*{Lecture 2}

Consider a right-travelling transverse wave:

\vspace*{10pt}
[Image]
\vspace*{10pt}

A particle on the string only has transverse velocity. So $\frac{\mathrm{d}y}{\mathrm{d}t}$ is not the speed of the wave. The same is true for longitudinal waves such as sound; the movement of the air molecules is vibrational.

\subsection*{Phase and Phase Difference}

\begin{align*}
u(t) = A \cos(kx - \omega t + \eta)
\end{align*}

where $\eta$ is the \textbf{phase of the wave}. $\eta$ is important for the superposition of waves, which leads to interference effects.

\vspace*{10pt}
[Image]
\vspace*{10pt}

\subsection*{Principle of Superposition}

\subsubsection*{Waves of the same frequency}

\begin{align*}
u_1 &= A_1 \cos(kx - \omega t) \\
u_2 &= A_2 \cos(kx - \omega t + \eta)
\end{align*}

Using the following trigonometric identity:

\begin{align*}
\cos(A) + \cos(B) = 2 \cos\left(\frac{A+B}{2}\right) \cos\left(\frac{A-B}{2}\right)
\end{align*}

If $\eta = 0$, $u_{total} = 2 u_1$ $\leftarrow$ \textbf{constructive interference}

If $\eta = \pi$, $u_{total} = 0$ $\leftarrow$ \textbf{destructive interference}

\subsubsection*{Waves of a similar frequency}

\begin{align*}
\text{Wave } & A \quad\quad k_1 = k + \Delta k \quad\quad \omega_1 = \omega + \Delta \omega \\
& B \quad\quad k_2 = k - \Delta k \quad\quad \omega_2 = \omega - \Delta \omega
\end{align*}

From this, we get:

\begin{align*}
u = 2 \cos(\Delta k x + \Delta \omega t) \cos(kx + \omega t)
\end{align*}

The first cosine in the above equation represents the `beat wave', and the second represents the `actual' wave.

Pianos have three strings per note, which are tuned to slightly different frequencies. Overall, when a note is played, we hear a note of frequency $\omega$ (an average of the different waves' frequencies), and a beat frequency $\Delta \omega$. If a piano had only two strings per note, and they were tuned to $254 \mathrm{Hz}$ and $258 \mathrm{Hz}$, we would hear a note at $256 \mathrm{Hz}$

\begin{align*}
\text{beat frequency } = 2 \Delta \omega
\end{align*}

\subsection*{Group Velocity}

There are two components to the total wave, with different speeds:

\begin{align*}
\text{the wave } \quad & v = \frac{\omega}{k} \\
\text{the beat wave } \quad & v = \frac{\Delta \omega}{\Delta k}
\end{align*}

As $\Delta \rightarrow 0$,

\begin{align*}
& v_g = \frac{\partial \omega}{\partial k}
\end{align*}

\end{document}