\documentclass[10pt,a4paper]{article}
\usepackage[utf8]{inputenc}
\usepackage{amsmath}
\usepackage{amsfonts}
\usepackage{amssymb}
\usepackage{graphicx}
\usepackage[margin=1.25in]{geometry}
\usepackage{hyperref}
\title{Introduction to Particle Physics}
\begin{document}

\part*{Lecture 2}

\section*{Units of Energy and Mass in EPP}

Charged particles are accelerated by placing them in an electric field.

The increase in the kinetic energy of a charge $q$ crossing a potential difference of $\Delta V$ is $\Delta E = -q \Delta V$ (in Joules). So for an electron charge $-e$, accelerated across a potential difference of $1 \textrm{V}$, the change in $E$ is:

\begin{align*}
\Delta E &= - (-e) \times 1 \\
&= 1.6 \times 10^{-19} \mathrm{J}
\end{align*}

This is the standard unit of energy in atomic physics.

\begin{equation}
1 \mathrm{eV} = 1.6 \times 10^{-19} \mathrm{J}
\end{equation}

We then have:

\begin{align*}
1 \mathrm{MeV} &= 1.6 \times 10^{-13} \mathrm{J} \text{ (nuclear physics)} \\
1 \mathrm{GeV} &= 1.6 \times 10^{-10} \mathrm{J} \text{ (standard in particle physics)}
\end{align*}

For masses, Einstein taught us:

\begin{align*}
E &= mc^2 \\
\rightarrow m &= \frac{E}{c^2}
\end{align*}

So usual to use as the unit of mass:

\begin{align*}
\frac{1 \mathrm{eV}}{c^2}
\end{align*}

\end{document}