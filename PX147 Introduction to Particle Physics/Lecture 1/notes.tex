\documentclass[10pt,a4paper]{article}
\usepackage[utf8]{inputenc}
\usepackage{amsmath}
\usepackage{amsfonts}
\usepackage{amssymb}
\usepackage{graphicx}
\usepackage[margin=1.25in]{geometry}
\title{Introduction to Particle Physics}
\begin{document}

\part*{Lecture 1}

\section*{The Guiding Principles of Elementary Particle Physics}

Elementary Particle Physics aims to describe the constituents of matter and the interactions among them in the simplest possible terms.

There are three guiding principles:

\begin{enumerate}
\item Composition
\item Symmetry
\item Unification
\end{enumerate}

By composition, we mean the principle that complex objects are composed of many simple things, bound together by forces or interactions. This helps explain them. But we need to understand the properties of the constituents and their interactions.

Symmetry is a regularity of something. Once we have the relevant parts of something with symmetry, the rest follows. It is the lack of detail or complexity. If something has spherical symmetry, I need only describe how it behaves as I move out from the centre in one direction. All other directions look identical. The laws of nature exhibit symmetries and are deeply related to conservation laws.

Unification is the idea that different physical phenomena can be described as different aspects or manifestations of the same fundamental principles or effects. Electromagnetism is a classic example.

Until 1887, electricity and magnetism were understood as largely different phenomena. In 1887, James Clarke Maxwell found that they were in fact different aspects of the same phenomenon. He produced a set of four coupled equations which together described 'electromagnetism' in an internally consistent way. As a by-product, he predicted for the first time the existence of electromagnetic waves travelling at the speed of light. They were discovered experimentally six years later by Heinrich Hertz.

Particle physicists are always looking for ways to unify the laws which describe the four fundamental interactions (strong, weak, electromagnetic, gravitational) between elementary particles. This is hoped to describe nature more simply.

\section*{Basic Building Blocks}

An elementary particle has no structure. It is as simple as it can be. It is point-like.

Take any molecule and it is composed of a number of atoms of different types. Take any atom, it has a nucleus and a number of electrons orbiting it.

If we take an electron, there is currently no evidence for any substructure to it. It exists at a point. All electrons are indistinguishable.

Each nucleus contains several protons and neutrons (except hydrogen). The only difference between atoms of different types is the number of protons, neutrons, and electrons. Atoms are electrically neutral so that the number of protons is equal to the number of electrons. What defines an element is just the number of protons. Inside protons and neutrons are smaller particles - called quarks, and gluons (generically called partons together). They are apparently point-like - no evidence for substructure.

\end{document}