\documentclass[10pt,a4paper]{article}
\usepackage[utf8]{inputenc}
\usepackage{amsmath}
\usepackage{amsfonts}
\usepackage{amssymb}
\usepackage{graphicx}
\usepackage[margin=1.25in]{geometry}
\usepackage{hyperref}

\newcommand{\gevcc}{\frac{\mathrm{GeV}}{c^2}}
\newcommand{\evcc}{\frac{\mathrm{eV}}{c^2}}

\title{Introduction to Particle Physics}
\begin{document}

\part*{Lecture 3: Antiparticles and Colour}

\section*{Antiparticles}

Every elementary particle type has a related anti-particle type. They have exactly the same mass but opposite electric charge and flavour.

\section*{Colour}

All quarks carry a 'colour': red, gree, or blue. Anti-quarks carry anti-colour. Not the same as optical colour, but is analogous. Colour is the source of the strong interaction.

\section*{Composites of Quarks - Baryons and Mesons}

There is a composition law or principle for quarks which states that any composite hadron must be colourless. There are two ways to achieve this:

\begin{enumerate}
\item Three quarks of any flavour, each of different colour - as in light: red + green + blue = white. Called a baryon. (e.g. proton, neutron)
\item A quark and an antiquark of any flavour but of equal and opposite colour. $\pi^+ = u\bar{d}$, $K^0 = d\bar{s}$. Called mesons.
\end{enumerate}

This composition principle has got a name: \textbf{confinement}. The origin of this name is that colour is confined inside hadrons $\rightarrow$ quarks themselves are also \textbf{confined} inside hadrons. They are elementary particles, but cannot exist as free particles. Colour can be passed from quark to quark. Particle-antiparticle pairs can be produced from the vacuum by the conversion of energy into mass. Most baryons and mesons carry flavour but not colour. Quarks heavier than $u$ and $d$ are unstable and decay, usually into lighter quarks.

\section*{Neutrinos}

Some nuclei are unstable, having too many neutrons. A neutron, under these circumstances, can transform itself into a proton. A $d$ quark changes into a $u$ quark and emits an electron and a light neutral particle called an antineutrino. This is beta decay.

[Image here]

\begin{align*}
n & \rightarrow p + e + \bar{\nu_e} \\
d & \rightarrow u + e + \bar{\nu_e}
\end{align*}

(A neutron does not 'contain' $p$, $e$, $\bar{\nu_e}$.)

When first discovered, these electrons were named $\beta$-particles before being identified as electrons.

$e$ and $\nu_e$ are examples of leptons (meaning 'light ones').

\textbf{N. B.} $\bar{\nu_e}$ was inferred because in $\beta$-decay the electron has a continuous spectrum (a frequency distribution of energy).

[Image here]

Without the $\bar{\nu_e}$, the final state, having only two particles, would show instead a single electron energy due to energy-momentum conservation.

[Image here]

With three particles in the final state, the continuous distribution is explained.

\section*{Leptonic Periodic Table and Lepton Flavour}

In 1987, a heavy copy of the electron was discovered. It had all the properties of the electron, except that it was about 200 times heavier. It was named the muon.

\begin{align*}
m_{\mu} = 0.105 \gevcc
\end{align*}

The muon is unstable, decaying with a lifetime of $\approx 2 \times 10^{-6} \mathrm{s}$ (to one electron and two neutrinos).

In 1962, the muon neutrino, $\nu_{\mu}$, was discovered - shown to be distinct from $\nu_e$.

In 1976, the tau lepton ($\tau$) was discovered, an even heavier copy of the electron, and in 2000, the tau neutrino was discovered, $\nu_{\tau}$.

The masses of the neutrinos are small ($< 1 \evcc$) but not zero.

Leptons have no colour, and do not bind together as quarks do. They do not 'feel' the strong force and do not interact with gluons.

\section*{Lepton Number}

There are three types of lepton number carried by leptons: electron number, muon number, and tauon number. These were thought to be conserved in all processes.

\end{document}