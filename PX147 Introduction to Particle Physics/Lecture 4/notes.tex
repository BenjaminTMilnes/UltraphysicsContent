\documentclass[10pt,a4paper]{article}
\usepackage[utf8]{inputenc}
\usepackage{amsmath}
\usepackage{amsfonts}
\usepackage{amssymb}
\usepackage{graphicx}
\usepackage[margin=1.25in]{geometry}
\usepackage{hyperref}

\newcommand{\gevcc}{\frac{\mathrm{GeV}}{c^2}}
\newcommand{\evcc}{\frac{\mathrm{eV}}{c^2}}
\newcommand{\nue}{\nu_{e}}
\newcommand{\numu}{\nu_{\mu}}
\newcommand{\nutau}{\nu_{\tau}}
\newcommand{\suh}{\uparrow {\left(\frac{1}{2}\right)}}
\newcommand{\sdh}{\downarrow {\left(-\frac{1}{2}\right)}}

\title{Introduction to Particle Physics}
\begin{document}

\part*{Lecture 4}

\begin{tabular}{c|c|c|c|c|c|c}
\hline
& $e$ & $\nue$ & $\mu$ & $\numu$ & $\tau$ & $\nutau$ \\
\hline
$e\#$ & 1 & 1 & 0 & 0 & 0 & 0 \\
\hline
$\mu\#$ & 0 & 0 & 1 & 1 & 0 & 0 \\
\hline
$\tau\#$ & 0 & 0 & 0 & 0 & 1 & 1 \\
\hline
\end{tabular}

with negative values for the antiparticles.

Recent discoveries of neutrino oscillations show that this is not exactly conserved, but they are a good approximation in most circumstances. Unless the neutrinos are propagating a long distance, oscillations are unlikely to happen.

However, the total lepton number:

\begin{align*}
L = N_{e} + N_{\mu} + N_{\tau}
\end{align*}

is exactly conserved in all processes. For all leptons $L = +1$ and for all antileptons $L = -1$.

Quarks have zero lepton number.

\section*{Fundamental Forces}

\subsection*{Spin-Fermions and Bosons}

Many elementary particles carry another quantity called spin.

The laws of quantum mechanics require that spin is quantised in $\frac{1}{2}\hbar$ where $\hbar = \frac{h}{2\pi}$. Spin has the dimensions of angular momentum, and can be thought of as intrinsic angular momentum. Think of the particle as spinning on its axis like a small rotating ball. It cannot be stopped; if it has spin it will always have it. This is even though they are point-like particles.

The behaviour of elementary particles depends on whether their spin is integer (0, 1, etc., in units of $\hbar$) or half-integer ($\frac{1}{2}$, $\frac{3}{2}$, ... etc.). Hence we discriminate between these two types of particles by the terms:

\begin{center}
fermions: $\frac{1}{2}$-integer spin \\
bosons: integer spin
\end{center}

Quarks and leptons all have spin $= \frac{1}{2} \hbar$
Gluons and photons have spin $= 1 \hbar$

The Higg's Boson has spin $= 0 \hbar$.
The graviton has spin $=2 \hbar$ (a particle of the gravitational field).

Non-elementary particles can also have spin:

Protons and neutrons have spin $= \frac{1}{2}$ as do many other baryons. Some baryons have spin $\frac{3}{2}$.

Angular momentum is a vector quantity. Spin is quantised in a particular direction (e.g. in the z-direction) and can only add along this direction. For a $q\bar{q}$ pair (a meson) - can have either:

\begin{align*}
\suh + \sdh = 0 \text{ (spin 0 meson)} \\
\suh + \suh = 1 \text{ (spin 1 meson)}
\end{align*}

For baryons, we can have:

\begin{align*}
\suh + \suh + \sdh = + \frac{1}{2} \text{ (spin }\frac{1}{2}\text{ baryon)} \\
\suh + \suh + \suh = + \frac{3}{2} \text{ (spin }\frac{3}{2}\text{ baryon)}
\end{align*}

Going back to elementary particles, for which spin-$\frac{1}{2}$ are matter constituents. Force-carrying particles are always bosons.

\subsection*{Particles and Force Mediators}

Classically, a force is proportional to a rate of change of momentum.

\begin{align*}
\textbf{F} &= m \frac{\mathrm{d}\textbf{v}}{\mathrm{d}t} \\
&=  \frac{\mathrm{d}\textbf{p}}{\mathrm{d}t}
\end{align*}

So that a force acting for a time produces a change of momentum, $\Delta p$.

In the quantum world, continuous processes do not occur and forces between particles act as single, discrete 'events', in which a third particle is exchanged between them, carrying momentum from one to the other.

Represented by a so-called Feynmann Diagram; a picture of what happens.

e.g. the repulsion between two electrons:

[Image here]

In this picture, the electrons repel each other by exchanging a photon $\gamma$. Think of $\gamma$ as being emitted by $e_{1}^{-}$, and absorbed by $e_{2}^{-}$. The photon momentum is:

\begin{align*}
\textbf{p}_{\gamma} = \textbf{p}_{2}^{\prime} -  \textbf{p}_{2} =  \textbf{p}_{1} -  \textbf{p}_{1}^{\prime}
\end{align*}

\textbf{N. B.} Momentum is conserved at both vertices (the emission and absorption point). Final overall momentum is equal to the initial overall momentum.

\begin{align*}
\textbf{p}_{1}^{\prime} + \textbf{p}_{2}^{\prime} = \textbf{p}_{1} + \textbf{p}_{2}
\end{align*}

As well as momentum, energy is also conserved:

\begin{align*}
E_{\gamma} = E_{2}^{\prime} - E_{2} = E_{1} - E_{1}^{\prime}
\end{align*}

Often such intermediate force-carrying particles are named 'virtual' particles. Force-carrying particles always have integer spin and are therefore bosons.

\end{document}